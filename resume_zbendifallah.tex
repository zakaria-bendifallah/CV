%%%%%%%%%%%%%%%%%%%%%%%%%%%%%%%%%%%%%%%%%
% Medium Length Professional CV
% LaTeX Template
% Version 2.0 (8/5/13)
%
% This template has been downloaded from:
% http://www.LaTeXTemplates.com
%
% Original author:
% Trey Hunner (http://www.treyhunner.com/)
%
% Important note:
% This template requires the resume.cls file to be in the same directory as the
% .tex file. The resume.cls file provides the resume style used for structuring the
% document.
%
%%%%%%%%%%%%%%%%%%%%%%%%%%%%%%%%%%%%%%%%%

%----------------------------------------------------------------------------------------
%	PACKAGES AND OTHER DOCUMENT CONFIGURATIONS
%----------------------------------------------------------------------------------------

\documentclass{resume} % Use the custom resume.cls style

\usepackage[left=0.75in,top=0.6in,right=0.75in,bottom=0.6in]{geometry} % Document margins

\usepackage[utf8]{inputenc}
\name{Zakaria Bendifallah} % Your name
\address{6A boulevard jean-jaures \\ Arpajon, 91290 - France} % Your address
\address{(+33)6~$\cdot$~95~$\cdot$~96~$\cdot$~42~$\cdot$~85 \\ zakaria.bendifallah@gmail.com} % Your phone number and email

\begin{document}


%----------------------------------------------------------------------------------------
%	Research Interest Section 
%----------------------------------------------------------------------------------------
\begin{rSection}{Research Interests}

\begin{rSubsection}{}{}{}{}
\item Performance analysis in HPC and related topics.
\item Tools and techniques for bottleneck detection.
\item Power consumption optimization.
%\item SW/HW codesign. 
\end{rSubsection}

%----------------------------------------------------------------------------------------
%	WORK EXPERIENCE SECTION
%----------------------------------------------------------------------------------------
\begin{rSection}{Professional Experience}

\begin{rSubsection}{Intel, Inc}{March 2016 - Present}{Software Engineer}{Meudon, France}
\item Work on energy consumption issues in HPC applications within the context 
      of the European project READEX. The project aims at exploiting the available dynamicity 
      within an application through profiling to save energy. It also focuses on exploring
      the possible hardware, system and application knobs which can be used to tune for energy.  
\item Explore energy tuning hardware level knobs on Intel micro-processors.
\item Establish an application domain knowledge formalism for the handling of energy tuning 
      application level knobs.         
\end{rSubsection}

\begin{rSubsection}{University of Versailles}{October 2015 - March 2016}{Post-doctoral researcher}{Versailles, France}
\item Design and test new performance analysis features for the Differential analysis approach for HPC applications.
\end{rSubsection}


\begin{rSubsection}{University of Versailles}{January 2013 - December 2015}{Teaching assistant}{Versailles, France}
\item Tutorials on compute kernel optimization for second year engineering students (3 years)
\item Tutorials on computer architecture for L2 students (one semester)  
\end{rSubsection}


\begin{rSubsection}{Intel, Inc}{October 2013 - January 2014}{Intern}{Champaign, Il, USA}
\item Explore the use of the Intel Processor Trace (PT) technology in the 
      study of the memory behaviour of compute intensive kernels.
\end{rSubsection}


\begin{rSubsection}{University of Versailles}{October 2011 - September 2015}
                   {Ph.D candidate}{Versailles, France}
\item Exploration and development of the Decremental Analysis approach, 
      an application performance analysis method which goal is to detect performance 
      pathologies within a loop by creating different modified versions
      of it. The primary target are hot compute intensive loops in scientific codes. 
      The method focuses primarily on core and memory sub-system related issues.
\item Upgrade and contribution to a binary rewriting tool called DECAN written in C
      and based on the MAQAO performance analysis framework. Contributions include
      new assembly transformations and parallel codes handling (~10K lines of code).   
\item Develop a strong experimental benchmarking methodology in LUA in order to 
      reduce measurement bias on small compute kernels. 
\end{rSubsection}

\newpage
\begin{rSubsection}{Exascale Computing Research center (ECR)}
                   {March 2011 - September 2011}{Intern}{Versailles, France}
\item Contribute to the development of a binary level application performance 
      analysis framework called MAQAO. The framework works by reverse engineering 
      the binary file of the application and creating an intermediate representation 
      (IR) in the form of a graph. The IR can be then used to build analysis tools.  
\item Integration of a fast and efficient loop detection algorithm to the core 
      MAQAO library.
\item Development of a floating-point exceptions detection module for MAQAO.
\end{rSubsection}


\begin{rSubsection}{Ecole Nationale Superieure d'Informatique (ESI)}
                   {September 2009 - July 2010}{Intern}{Algiers, Algeria}
\item Design and development of a client-server multi-modal application for web 
      content editing. The application enables database interrogation through a 
      handheld device with either a graphical user interface (GUI) or voice.
\item Work on voice enabled technologies using the VoiceXML standard and 
      Interactive Voice Response (IVR) solutions.
\item Development of a mobile application on the Android platform (V 1.5).  
\end{rSubsection}

\end{rSection}
%----------------------------------------------------------------------------------------
%	EDUCATION SECTION
%----------------------------------------------------------------------------------------

\begin{rSection}{Education}

{\bf University of Versailles, Versailles - France} \hfill {\em October 2011 - September 2015} \\ 
Ph.D in Computer Science\\
Thesis title: Generalization of the Decremental Performance Analysis to Differential Analysis

{\bf University of Versailles, Versailles - France} \hfill {\em September 2010 - September 2011} \\ 
M.S in High Performance Computing\\
Thesis title: Binary Performance Analysis Challenges

{\bf Ecole Nationale Superieure d'Informatique (ESI), Algiers - Algeria} \hfill {\em September 2005 - July 2010} \\ 
State Engineer in Computer systems\\
Thesis title: Program Update Through Handheld Devices 
\end{rSection}


%------------------------------------------------

%\begin{rSubsection}{AJAX Hosting}{December 2009 - October 2010}{Lead Developer}{Austin, TX}
%\item Aenean ut gravida lorem. Ut turpis felis, Perl pulvinar a semper sed, adipiscing id dolor.
%\item Curabitur dapibus enim sit amet elit pharetra tincidunt website feugiat nisl imperdiet. Ut convallis AJAX libero in urna ultrices accumsan.
%\item Cum sociis natoque penatibus et magnis dis MySQL parturient montes, nascetur ridiculus mus.
%\item In rutrum accumsan ultricies. Mauris vitae nisi at sem facilisis semper ac in est.
%\item Nullam cursus suscipit nisi, et ultrices justo sodales nec. Fusce venenatis facilisis lectus ac semper.
%\end{rSubsection}

%------------------------------------------------

%\begin{rSubsection}{TinySoft}{January 2008 - April 2010}{Web Designer \& Developer}{Gainesville, GA}
%\item Vivamus PostgreSQL fermentum semper porta. Nunc diam velit PHP, adipiscing ut tristique vitae
%\item Maecenas convallis ullamcorper ultricies stylesheets.
%\item Quisque mi metus, unit tests CSS ornare sit amet fermentum et, tincidunt et orci.
%\item Curabitur venenatis pulvinar tellus gravida ornare. Sed et erat faucibus nunc euismod ultricies ut id
%\end{rSubsection}

\end{rSection}

%----------------------------------------------------------------------------------------
%	TECHNICAL STRENGTHS SECTION
%----------------------------------------------------------------------------------------

\begin{rSection}{Technical Strengths}

\begin{tabular}{ @{} >{\bfseries}l @{\hspace{6ex}} l }
Computer Languages & C, C++, Lua, Python, Assembly (x86), Java, Linux Shell, \\
                   & HTML, CSS, Javascript\\
Protocols & XML, VoiceXML, JSON \\
Databases & MySQL \\
Programming paradigms & OpenMP, Pthread\\
Frameworks & Django, JQuery\\
Tools & Vim, Git, Latex, Slurm, Vtune, Likwid, Score-P \\
Platforms & GNU/Linux distributions , Android \\ 
\end{tabular}

\end{rSection}

%----------------------------------------------------------------------------------------
%	Languages
%----------------------------------------------------------------------------------------

\begin{rSection}{Languages}
\begin{itemize}
\item Arabic: mother tongue
\item French: fluent
\item English: fluent
\end{itemize}
\end{rSection}


%----------------------------------------------------------------------------------------
%	Miscellaneous
%----------------------------------------------------------------------------------------

\begin{rSection}{Miscellaneous}
\begin{itemize}
\item \textbf{Sports}: football, basketball, climbing
\item \textbf{Others}: reading, traveling 
\end{itemize}

\end{rSection}



\begin{rSection}{Publications / Conference papers}
\begin{itemize}
\item Koliai, Souad, Zakaria Bendifallah, Mathieu Tribalat, Cédric Valensi, Jean-Thomas Acquaviva, and William Jalby. "Quantifying performance bottleneck cost through differential analysis." In Proceedings of the 27th international ACM conference on International conference on supercomputing, pp. 263-272. ACM, 2013\item Bendifallah, Zakaria, William Jalby, José Noudohouenou, Emmanuel Oseret, Vincent Palomares, and Andres Charif Rubial. "Pamda: Performance assessment using maqao toolset and differential analysis." In Tools for High Performance Computing 2013, pp. 107-127. Springer International Publishing, 2014.
\item Wong, David C., V. Palomares, E. Oseret, Zakaria Bendifallah, Mathieu Tribalat, W. Jalby, and D. J. Kuck. "Vp3: A vectorization potential performance prototype." ser. WPMVP 15 (2015): 7.
\item J. Schuchart, M. Gerndt, P. G. Kjeldsberg, M. Lysaght, D. Horak, L. Riha, A. Gocht, M. Sourouri, M. Kumaraswamy, A. Chowdhury, M. Jahre, K. Diethelm, O. Bouizi, U. S. Mian, J. Kružík, R. Sojka, M. Beseda, V. Kannan, Z. Bendifallah, D. Hackenberg, and W. E. Nagel, “The READEX formalism for automatic tuning for energy efficiency”, Computing, 2017.
\end{itemize}    
\end{rSection}


%----------------------------------------------------------------------------------------
%	EXAMPLE SECTION
%----------------------------------------------------------------------------------------

%\begin{rSection}{Section Name}

%Section content\ldots

%\end{rSection}

%----------------------------------------------------------------------------------------

\end{document}
