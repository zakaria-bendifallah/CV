%%%%%%%%%%%%%%%%%%%%%%%%%%%%%%%%%%%%%%%%%
% Medium Length Professional CV
% LaTeX Template
% Version 2.0 (8/5/13)
%
% This template has been downloaded from:
% http://www.LaTeXTemplates.com
%
% Original author:
% Trey Hunner (http://www.treyhunner.com/)
%
% Important note:
% This template requires the resume.cls file to be in the same directory as the
% .tex file. The resume.cls file provides the resume style used for structuring the
% document.
%
%%%%%%%%%%%%%%%%%%%%%%%%%%%%%%%%%%%%%%%%%

%----------------------------------------------------------------------------------------
%	PACKAGES AND OTHER DOCUMENT CONFIGURATIONS
%----------------------------------------------------------------------------------------

\documentclass{resume} % Use the custom resume.cls style

\usepackage[left=0.75in,top=0.6in,right=0.75in,bottom=0.6in]{geometry} % Document margins

\name{Zakaria Bendifallah} % Your name
\address{27 rue Auguste Blanche \\ Puteaux, 92800 - France} % Your address
\address{(+33)6~$\cdot$~95~$\cdot$~96~$\cdot$~42~$\cdot$~85 \\ zakaria.bendifallah@gmail.com} % Your phone number and email

\begin{document}

%----------------------------------------------------------------------------------------
%	WORK EXPERIENCE SECTION
%----------------------------------------------------------------------------------------

\begin{rSection}{Experience}

\begin{rSubsection}{University of Versailles}{October 2015 - Present}{Post-doctoral researcher}{Versailles, France}
\item  
\item 
\end{rSubsection}


\begin{rSubsection}{Intel, Inc}{October 2013 - January 2014}{Intern}{Champaign, Il, USA}
\item Explored the use of Intel Processor Trace (PT) technology for the 
      study of the memory behaviour of compute intensive kernels
\end{rSubsection}


\begin{rSubsection}{University of Versailles}{October 2011 - September 2015}
                   {Ph.D candidate}{Versailles, France}
\item Exploraiton of the potential of an application performance analysis approach
      called Decremental Analysis. The approach is comparative, and aims to detect
      performance pathologies within a loop by creating different modified versions
      of it. The primary target are hot loops of simulation codes. The method focuses 
      primarily on core and memory sub-system related issues.
\item Upgrade and contribution to a binary rewriting tool called DECAN written in C
      and based on the MAQAO performance analysis framework. Contributions include
      new assembly transformations and parallel codes handling.   
\item Develop a strong experimental benchmarking methodology in LUA in order to 
      reduce measurement bias on small compute kernels. 
\end{rSubsection}


\begin{rSubsection}{Exascale Computing Research center (ECR)}
                   {March 2011 - September 2011}{Intern}{Versailles, France}
\item Contribute to the development of a binary level application performance 
      analysis framework called MAQAO. The framework works by reverse engineering 
      the binary file of the application and creating an intermediate representation 
      (IR) in the form of a graph (the nodes are basic blocks and arcs represent 
      control flow). The IR can be then used to build analysis tools.  
\item Integration of a fast and efficient loop detection algorithm in the core 
      library of MAQAO.  
\item Development of a floating-point exceptions detection module for MAQAO.
\end{rSubsection}


\begin{rSubsection}{Ecole Nationale Superieure d'Informatique (ESI)}
                   {September 2009 - July 2010}{Intern}{Algiers, Algeria}
\item Design and development of a client-server multi-modal application for web 
      content editing. The application enables database interrogation through a 
      handheld device with either a graphical user interface (GUI) or voice.
\item Worked on voice enabled technologies using the VoiceXML standard and 
      Interactive Voice Response (IVR) solutions.
\item Development of a mobile application on the Android platform (V 1.5).  
\end{rSubsection}

%----------------------------------------------------------------------------------------
%	EDUCATION SECTION
%----------------------------------------------------------------------------------------

\begin{rSection}{Education}

{\bf University of Versailles, Versailles - France} \hfill {\em October 2011 - September 2015} \\ 
Ph.D in Computer Science\\
Thesis title: Generalization of the Decremental Performance Analysis to Differential Analysis

{\bf University of Versailles, Versailles - France} \hfill {\em September 2010 - September 2011} \\ 
M.S in High Performance Computing\\
Thesis title: Binary Performance Analysis Challenges

{\bf Ecole Nationale Superieure d'Informatique (ESI), Algiers - Algeria} \hfill {\em September 2005 - July 2010} \\ 
State Engineer in Computer systems\\
Thesis title:  
\end{rSection}


%------------------------------------------------

%\begin{rSubsection}{AJAX Hosting}{December 2009 - October 2010}{Lead Developer}{Austin, TX}
%\item Aenean ut gravida lorem. Ut turpis felis, Perl pulvinar a semper sed, adipiscing id dolor.
%\item Curabitur dapibus enim sit amet elit pharetra tincidunt website feugiat nisl imperdiet. Ut convallis AJAX libero in urna ultrices accumsan.
%\item Cum sociis natoque penatibus et magnis dis MySQL parturient montes, nascetur ridiculus mus.
%\item In rutrum accumsan ultricies. Mauris vitae nisi at sem facilisis semper ac in est.
%\item Nullam cursus suscipit nisi, et ultrices justo sodales nec. Fusce venenatis facilisis lectus ac semper.
%\end{rSubsection}

%------------------------------------------------

%\begin{rSubsection}{TinySoft}{January 2008 - April 2010}{Web Designer \& Developer}{Gainesville, GA}
%\item Vivamus PostgreSQL fermentum semper porta. Nunc diam velit PHP, adipiscing ut tristique vitae
%\item Maecenas convallis ullamcorper ultricies stylesheets.
%\item Quisque mi metus, unit tests CSS ornare sit amet fermentum et, tincidunt et orci.
%\item Curabitur venenatis pulvinar tellus gravida ornare. Sed et erat faucibus nunc euismod ultricies ut id
%\end{rSubsection}

\end{rSection}

%----------------------------------------------------------------------------------------
%	TECHNICAL STRENGTHS SECTION
%----------------------------------------------------------------------------------------

\begin{rSection}{Technical Strengths}

\begin{tabular}{ @{} >{\bfseries}l @{\hspace{6ex}} l }
Computer Languages & C, LUA, Assembly (x86), Java, BASH \\
Protocols & XML, VoiceXML \\
Databases & MySQL \\
Tools & Vim, Git, latex, Slurm \\
Platforms & Android \\ 
\end{tabular}

\end{rSection}

%----------------------------------------------------------------------------------------
%	Teachings
%----------------------------------------------------------------------------------------

\begin{rSection}{Teachings}


\end{rSection}


%----------------------------------------------------------------------------------------
%	EXAMPLE SECTION
%----------------------------------------------------------------------------------------

%\begin{rSection}{Section Name}

%Section content\ldots

%\end{rSection}

%----------------------------------------------------------------------------------------

\end{document}
